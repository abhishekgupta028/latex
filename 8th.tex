\documentclass[]{report}
\usepackage{hyperref}
\usepackage{draftwatermark}
\SetWatermarkText{EXP 8 RESULT}
\SetWatermarkScale{2}
\usepackage[english]{babel}
\usepackage{amsthm}
\usepackage{amsmath} % for correct math environments

\newtheorem{theorem}{Theorem}[section]
\newtheorem{corollary}{Corollary}[theorem]
\newtheorem{lemma}[theorem]{Lemma}
\theoremstyle{definition}
\newtheorem{definition}{Definition}[section]

\begin{document}

\begin{theorem}
Let \(f\) be a function whose derivative exists at every point, then \(f\) is a continuous function.
\end{theorem}

\begin{theorem}[Pythagorean theorem]
\label{pythagorean}
This is a theorem about right triangles and can be summarised in the next equation:
\[
x^2 + y^2 = z^2
\]
\end{theorem}

And a consequence of Theorem~\ref{pythagorean} is the statement in the next corollary.

\begin{corollary}
There’s no right triangle whose sides measure 3cm, 4cm, and 6cm.
\end{corollary}

You can reference theorems such as \ref{pythagorean} when a label is assigned.

\begin{lemma}
Given two line segments whose lengths are \(a\) and \(b\) respectively, there is a real number \(r\) such that \(b = ra\).
\end{lemma}

\begin{definition}[Absolute value function]
The absolute value function can be specified as a two-part definition as follows:
\[
|x| =
\begin{cases}
x & \text{if } x \geq 0 \\
-x & \text{if } x < 0
\end{cases}
\]
\end{definition}


\end{document}
